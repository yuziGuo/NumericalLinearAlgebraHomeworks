\section*{Question 6}
\textcolor{blue}{[QR Factorization]} Consider again the matrices A and B of question 6.


(a)  Using any method you like, determine (on paper) a reduced QR factorization $A = \hat{Q}\hat{R}$ and a full QR factorization $A = QR$.


(b)  Again using any method you like, determine reduced and full QR factorizations $B = \hat{Q}\hat{R}$ and $B = QR$.

% \begin{align*}
    % A = \begin{bmatrix}
    %     1 & 0\\ 0 & 1\\ 1 & 0
    % \end{bmatrix},\quad
%     B = \begin{bmatrix}
%         1 & 2\\ 0 & 1\\ 1 & 0
%         \end{bmatrix}
%     \end{align*}

\subsection*{Answer}

(a1)\textcolor{blue}{[Reduced QR factorization of $A$]}. 
Denote the reduced QR factorization of $A$ as
$A = \hat{Q}\hat{R}$, $
    \hat{Q} \in \mathbb{R}^{3 \times 2},
    \hat{R} \in \mathbb{R}^{2 \times 2} 
    $,

\begin{align*}
    &A = [\mathbf{a_1}, \mathbf{a_2}],
    \quad 
    \hat{Q} = [\mathbf{q_1}, \mathbf{q_2}], \quad
    \hat{R} = \begin{bmatrix}
                r_{11} & r_{12}\\ 0 & r_{22}
            \end{bmatrix},\\
    &\mathbf{a_1, a_2} \in \mathbb{R}^{3\times 1}, \quad
    \mathbf{q_1, q_2} \in \mathbb{R}^{3\times 1},\\
    & r_{11}, r_{12}, r_{22} \in \mathbb{R}.
\end{align*}

As been given, $\mathbf{a_1} \bot \mathbf{a_2}$,
so we directly derive that:
% \begin{equation}
%     \mathbf{q_1} = \mathbf{a_1}/\|\mathbf{a_1}\|, \quad
%     \mathbf{q_2} = \mathbf{a_2}/\|\mathbf{a_2}|
% \end{equation}

\begin{align}
    \label{eq:6.1}
    \mathbf{q_1} &= \mathbf{a_1}/\|\mathbf{a_1}\|, \quad r_{11} = \|\mathbf{a_1}\|\\
    \label{eq:6.2}
    \mathbf{q_2} &= \mathbf{a_2}/\|\mathbf{a_2}\|, \quad r_{22} = \|\mathbf{a_2}\|
\end{align},
and 
\begin{align}
    \mathbf{a_2} &= r_{12}\mathbf{q_1} + r_{22}\mathbf{q_2}
    % \Rightarrow r_{12}\mathbf{q_1} + r_{22}\mathbf{q_2} = \mathbf{a_2}
    \nonumber \\
    \Rightarrow
    \langle\mathbf{a_2}, \mathbf{q_1} \rangle 
    &=  
    r_{12} \langle\mathbf{q_1}, \mathbf{q_1} \rangle 
    + r_{22}\langle\mathbf{q_2}, \mathbf{q_1} \rangle 
    \nonumber \\
    \label{eq:6.3}
    \Rightarrow r_{12} &= 0.
\end{align}

Compounding equations \eqref{eq:6.1} \eqref{eq:6.2} and \eqref{eq:6.3}, 
we get we the reduced QR factorization of $A$:
\begin{align*}
    \hat{Q} = \begin{bmatrix}
        \sqrt{2}/2 & 0\\ 0 & 1\\ \sqrt{2}/2 & 0
    \end{bmatrix},\quad
    \hat{R} = \begin{bmatrix}
        \sqrt{2} & 0\\ 0 & 1
        \end{bmatrix}
    \end{align*}


(a2)\textcolor{blue}{[Full QR factorization of $A$]}. 
Denote the full QR factorization of $A$ as
$A = QR$, $
    Q \in \mathbb{R}^{3 \times 3},
    R \in \mathbb{R}^{3 \times 2} 
    $,
note that in the equations below, 
all the values except for $\mathbf{q_3}$
keep the same with in reduced QR factorization:

\begin{align*}
    A = [\mathbf{a_1}, \mathbf{a_2}],
    \quad 
    Q = [\mathbf{q_1}, \mathbf{q_2},\textcolor{red}{\mathbf{q_3}}], \quad
    R = \begin{bmatrix}
                r_{11} & r_{12} 
                \\ 0 & r_{22} 
                \\ 0 & 0 
            \end{bmatrix},\\
    % \mathbf{a_1, a_2, q_1, q_2} \in \mathbb{R}^{2\times 1},
    % \quad  r_{11}, r_{12}, r_{22} \in \mathbb{R}.
\end{align*}

As $\mathbf{q_3} \bot \mathbf{q_1}$ and 
$\mathbf{q_3} \bot \mathbf{q_2} $, 
$\mathbf{q_3} \in Null(\hat{Q})$,


\begin{align*}
    \hat{Q}^T\mathbf{q_3} &= 0, \quad \|\mathbf{q_3}\| = 1 \nonumber \\
    \Rightarrow&
    \mathbf{q_3} = \pm [\sqrt{2}/2, 0, -\sqrt{2}/2]^T
\end{align*}

Let $\mathbf{q_3} = [\sqrt{2}/2, 0, -\sqrt{2}/2]^T$, we get
the full QR factorization of $A$
:
\begin{align*}
    Q = \begin{bmatrix}
        \sqrt{2}/2 & 0 & \sqrt{2}/2 \\ 0 & 1 & 0\\ \sqrt{2}/2 & 0 & -\sqrt{2}/2\
    \end{bmatrix},\quad
    R = \begin{bmatrix}
        \sqrt{2} & 0\\ 0 & 1 \\ 0 & 0
        \end{bmatrix}
    \end{align*}



(b1)\textcolor{blue}{[Reduced QR factorization of $B$]}. 
Denote the reduced QR factorization of $B$ as
$B = \hat{Q}\hat{R}$, $
    \hat{Q} \in \mathbb{R}^{3 \times 2},
    \hat{R} \in \mathbb{R}^{2 \times 2} 
    $,

\begin{align*}
    &B = [\mathbf{b_1}, \mathbf{b_2}],
    \quad 
    \hat{Q} = [\mathbf{q_1}, \mathbf{q_2}], \quad
    \hat{R} = \begin{bmatrix}
                r_{11} & r_{12}\\ 0 & r_{22}
            \end{bmatrix},\\
    &\mathbf{b_1, b_2}  \in \mathbb{R}^{3\times 1}, 
    \mathbf{q_1, q_2} \in \mathbb{R}^{3\times 1}\\
    &r_{11}, r_{12}, r_{22} \in \mathbb{R}.
\end{align*}

By the inherent property of QR factorization, 

\begin{align*}
    &\mathbf{q_1} \bot \mathbf{q_2},\\
    &span[\mathbf{q_1}] = span[\mathbf{b_1}],\\
    &span[\mathbf{q_1, q_2}] = span[\mathbf{b_1, b_2}]    
\end{align*}
we derive that:

\begin{align}
    \mathbf{q_1} &= \mathbf{b_1}/\|\mathbf{b_1}\|, \quad r_{11} = \|\mathbf{b_1}\|
    \nonumber \\
    \Rightarrow
    \label{eq:6.4} 
    \mathbf{q_1} &= [\sqrt{2}/2, 0, \sqrt{2}/2]^T, \quad r_{11} = \sqrt{2} 
\end{align},

and 
\begin{align}
    \label{eq:6.5}
    \mathbf{b_2} &= r_{12}\mathbf{q_1} + r_{22}\mathbf{q_2}
    \\
    \Rightarrow
    \langle\mathbf{b_2}, \mathbf{q_1} \rangle 
    &=  
    r_{12} \langle\mathbf{q_1}, \mathbf{q_1} \rangle 
    + r_{22}\langle\mathbf{q_2}, \mathbf{q_1} \rangle 
    \nonumber \\
    \label{eq:6.6}
    \Rightarrow r_{12} &= \langle\mathbf{b_2}, \mathbf{q_1} \rangle =  \sqrt{2}.
\end{align}

Feed \eqref{eq:6.6} into \eqref{eq:6.5}, and apply $\|\mathbf{q_2}\| = 1$we get:
\begin{align}
    r_{22}\mathbf{q_2} &= \mathbf{b_2} - r_{12}\mathbf{q_1} = [1, 1, -1]^T
    \nonumber \\
    \label{eq:6.7}
    &\Rightarrow r_{22} = \sqrt{3}, \mathbf{q_2} = [\sqrt{3}/3, \sqrt{3}/3, -\sqrt{3}/3]
\end{align}

Compounding equations \eqref{eq:6.4} to \eqref{eq:6.7}, 
we get we the reduced QR factorization of $B$:
\begin{align*}
    B = \hat{Q}\hat{R}, \quad
    \hat{Q} = \begin{bmatrix}
        \sqrt{2}/2 & \sqrt{3}/3\\ 0 & \sqrt{3}/3\\ \sqrt{2}/2 & -\sqrt{3}/3
    \end{bmatrix},\quad
    \hat{R} = \begin{bmatrix}
        \sqrt{2} & \sqrt{2} \\ 0  & \sqrt{3}
        \end{bmatrix}.
    \end{align*}





(b2)\textcolor{blue}{[Full QR factorization of $B$]}. 
Denote the full QR factorization of $B$ as
$B = QR$, $
    Q \in \mathbb{R}^{3 \times 3},
    R \in \mathbb{R}^{3 \times 2} 
    $,
note that in the equations below, 
all the values except for $\mathbf{q_3}$
keep the same with in reduced QR factorization:

\begin{align*}
    B = [\mathbf{b_1}, \mathbf{b_2}],
    \quad 
    Q = [\mathbf{q_1}, \mathbf{q_2},\textcolor{red}{\mathbf{q_3}}], \quad
    R = \begin{bmatrix}
                r_{11} & r_{12} 
                \\ 0 & r_{22} 
                \\ 0 & 0 
            \end{bmatrix},\\
    % \mathbf{a_1, a_2, q_1, q_2} \in \mathbb{R}^{2\times 1},
    % \quad  r_{11}, r_{12}, r_{22} \in \mathbb{R}.
\end{align*}

As $\mathbf{q_3} \bot \mathbf{q_1}$ and 
$\mathbf{q_3} \bot \mathbf{q_2} $, 
$\mathbf{q_3} \in Null(\hat{Q})$,

\begin{align*}
    \hat{Q}^T\mathbf{q_3} &= 0, \quad \|\mathbf{q_3}\| = 1 \nonumber \\
    \Rightarrow&
    \mathbf{q_3} = \pm [-\sqrt{6}/6, \sqrt{6}/3, \sqrt{6}/6]^T
\end{align*}

Let $\mathbf{q_3} = [-\sqrt{6}/6, \sqrt{6}/3, \sqrt{6}/6]^T$, we get
the full QR factorization of $B$
:
\begin{align*}
    B = QR, \quad
    Q = \begin{bmatrix}
        \sqrt{2}/2 & \sqrt{3}/3 & -\sqrt{6}/6\\ 0 & \sqrt{3}/3 & \sqrt{6}/3\\ \sqrt{2}/2 & -\sqrt{3}/3 & \sqrt{6}/6
    \end{bmatrix},\quad
    R = \begin{bmatrix}
        \sqrt{2} & \sqrt{2}  \\ 0  & \sqrt{3}  \\ 0 & 0
        \end{bmatrix}.
    \end{align*}
