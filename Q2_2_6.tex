\section*{Question 2}
\textcolor{blue}{[Rank-one perturbation of the identity]}
2.6. If $u$ and $v$ are m-vectors, 
the matrix $A = I + uv*$ is known as a rank-one perturbation of the identity. 
\begin{itemize}
    \item Show that if $A$ is nonsingular, then its inverse has the form $A^{-1} = I + auv*$ for some scalar $a$, and give an expression for $a$.
    \item For what $u$ and $v$ is $A$ singular? If it is singular, what is $null (A)$?
\end{itemize}



\subsection*{The first part of proof}
Suppose that $A$ is non-singular, 
let $B = I + auv^*$, we will prove that for any $A$,
we can find a corresponding $a$, and thus a corresponding $B$,
so that $B = A^{-1}$.

If $AB = I$, we get:
\begin{align*}
    AB = I &\rightarrow (I + uv^*)(I + auv^*) = I\\
           &\rightarrow I + auv^* + uv^* + au(v^*u)v* = I \\
           &\rightarrow (a + 1 + av^*u)uv^* = 0 \\
           &\rightarrow a(1 + v^*u) + 1 = 0 \ or\ uv^* = 0
\end{align*}

% which indicates that $uv^* = 0$ or $(a + 1 +v^*u)=0$. 
As been given,  $rank(u) = rank(v) = 1 > 0$,
so the $a$ we need satisfies:
\begin{equation}
    \label{eq:1.1}
    a(1 + v^*u) = -1
\end{equation}

Next we show that $v^*u \neq -1$. Suppose that $v^*u = -1$, then 
$Au = (I + uv^*)u = u + u*(-1) =  \mathbf{0}, $ conflicting that
$u\neq \mathbf{0}$ and $A$ is non-singular.

So from \eqref{eq:1.1} we further get:
\begin{equation}
a = - \dfrac{1}{1+v^*u}
\end{equation}

\subsection*{The second part of proof}

In this part, we show that $v^*u = -1$ is the necessary and sufficient condition
when $A$ is singular.

\subsubsection*{sufficiency}
As been shown in the former part,
when $v^*u = -1$,
$Au = (I + uv^*)u = u + u*(-1) =  \mathbf{0}$,
then $A$ is singular.

\subsubsection*{necessity}
Once $v^*u \neq -1$, according to the former part,
we can construct a matrix $ B = I - \dfrac{1}{1+ v^*u}uv^*$
that $AB = I$, which indicates that $A$ is non-singular.
So we have proved the necessity.

\subsubsection*{The Null Space of A}
\begin{align*}
    Ax &= 0 \\
    \rightarrow (I + uv^*)x &= 0 \\
    \rightarrow x &= - (v^*x)u \\
    \rightarrow x & // u
    \rightarrow Null(A) = span([u])
\end{align*}





