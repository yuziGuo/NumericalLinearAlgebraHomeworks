\section*{Question 1}
\textcolor{blue}{[Range, Linear Mapping and Matrix]}
Let $f_1, ..., f_8$ be a set of functions defined on the interval $[1, 8]$ 
with the property that for any numbers $d_1, ..., d_8$, 
there exists a set of  coefficients $c_1, ..., c_8$ such that
 \begin{equation*}
     \sum_{j=1}^8 c_jf_j(i) = d_i,\quad i=1, ..., 8
 \end{equation*}
(a) Show by appealing to the theorems of this lecture 
that $d_1, ..., d_8$ determine $c_1, ..., c_8$ uniquely.
\\
(b)  Let $A$ be the $8 \times 8$ matrix representing the linear mapping 
from data $d_1, ..., d_8$ to coefficients $c_1, ..., c_8$.
What is the $i$, $j$ entry of $A^{-1}$ ?


\subsection*{Answer}
% \subsection*{Proof}
(a). 
For proof, we will use two theorems from APPLIED NUMERICAL
ANALYSIS below:
\begin{theorem}
\label{theorem:range}
Range($A$) is the space spanned by the columns of $A$.
\end{theorem}

\begin{theorem}
\label{theorem:fullrank}
A matrix $A \in \mathbb{C}^{m \times n}$ has full rank 
if and only if maps no two distinct vectors to the same vector.
\end{theorem}

\begin{proof}
    Denote $F = \begin{bmatrix}
        f_1(1) & ... & f_8(1)\\
        ...   & ... & ... \\
        f_1(8) & ... & f_8(8)\\
    \end{bmatrix}$.
    We can re-write the mapping $f$ on $c_1, ..., c_8$ to 
    $d_1, ..., d_8$ in a matrix-vector-product way:
    \begin{equation*}
        \label{eq:q1}
        F
        \begin{bmatrix}
            c_1 \\ ... \\ c_8 \\
        \end{bmatrix} 
        = 
        \begin{bmatrix}
            d_1 \\ ... \\ d_8 
        \end{bmatrix} .
    \end{equation*}

    Note that the value of $d_i$ can be any number, 
    and we can view $[d_i]_{i\in[1,8]}$ to be spanned by
    the columns of $F$, so $Range(F) = 8$. By applying 
    Theorem \ref{theorem:range}, we get that $F$ is full-rank.
    
    So, $d_1, ..., d_8$ determine $c_1, ..., c_8$ uniquely.
    Otherwise, we'll find two different vectors that map 
    the columns of full-rank matrix $F$ to a same vector, 
    which conflicts Theorem \ref{theorem:fullrank}. 
    Until now We have finished sub-question (a).
\end{proof}

(b). As $F$ is full-rank, \eqref{eq:q1} can be written as:
\begin{equation*}
    \label{eq:q1}
    \begin{bmatrix}
        c_1 \\ ... \\ c_8 \\
    \end{bmatrix} 
    = 
    F^{-1}
    \begin{bmatrix}
        d_1 \\ ... \\ d_8 
    \end{bmatrix} .
\end{equation*}

So $A = F^{-1} \rightarrow A^{-1} = F 
\rightarrow A^{-1}_{i,j} = F_{ij} = f_j(i)$.
Until now We have finished sub-question (b).




