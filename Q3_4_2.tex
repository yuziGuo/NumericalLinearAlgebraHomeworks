\section*{Question 3}
\textcolor{blue}{[Singular Values of Rotated Matrix]} Suppose $A$ is an $m \times n$ matrix and $B$ is the $n \times m$ matrix obtained by
rotating $A$ ninety degrees clockwise on paper (not exactly a standard mathematical transformation!). 
Do $A$ and $B$ have the same singular values? 
Prove that the answer is yes or give a counterexample.

\subsection*{Answer}
Yes. 

\subsection*{Proof}
\begin{notation}
We use $A^{R}$ as a  notation for the  ninety-degree clock-wisely rotated $A$,
and $A^T$ for the transpose of $A$.
\end{notation}

For proof, we will use the theorem and corollary below:
\begin{theorem}
\label{theorem:singular}
The nonzero singular values of A are the square roots of the nonzero eigenvalues of $A*A$ or $AA*$.[NLA Theorem 5.4]
\end{theorem}

We will also use the immediate corollary of \ref{theorem:singular} below:
\begin{corollary}
\label{corollary:singular}
For any matrix  $M \in \mathbb{R}^{m \times n}$,
$AA^*$ and $A^*A$ share the same singular values.
\end{corollary}

\begin{lemma}[Relation between $A^{R}$ and $A^T$]
\label{lem:lem1}
For any matrix $A \in \mathbb{R}^{m \times n}$,
% $A^T, A^R \in \mathbb{R}^{n \times m}$,
$A^T_{i,j}=A^R_{i,m-j-1}$ 
\end{lemma}

\begin{proof}
$A^T_{i,j} = A{j,i} = A^R_{i,m-j-1}$.
\end{proof}

\begin{lemma}[Singular value of $A^{R}$ and $A^T$]
\label{lem:lem2}
$A^{R}$ and $A^T$ share the same singular values.
\end{lemma}

\begin{proof}
Denote $A^T$ to be $C$,  $A^R$ to be $D$.
According to \ref{lem:lem1},
the elements in each row $i$ of $C$ and $D$ are the same,
they are just permuted (more precisely, permuted in reversed orders), and the permutation are parallel among each row. So the inner product of any two rows in $C$ and $D$ are the same:
\begin{equation*}
    \langle C[i,:], C[j,:] \rangle = \langle D[i,:], D[j,:] \rangle
\end{equation*}

So: 
\begin{equation}
    \label{eq:cct}
    CC^T = [\langle C[i,:], C[j,:] \rangle]_{ij} = 
    \langle D[i,:], D[j,:] \rangle = DD^T
\end{equation}

By Theorem \ref{theorem:singular} and \eqref{eq:cct} we derive that $C$ and $D$ share the same singular values . Thus, we have finished the proof of Lemma \ref{lem:lem2}.
\end{proof}

\begin{proof}
Back to the question. By \ref{lem:lem2}, $B$ shares the same singular values with $A^T$; By \ref{corollary:singular}, $A^T$ shares the same singular values with $A$. So  $A$ and $B$ have the same singular values.
\end{proof}



